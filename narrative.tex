%\lhead{\sffamily  {\small Dark Energy}}
\lhead{{\small LSST undergraduate internships at Fermilab}}

\section{LSST undergraduate internships at Fermilab}
%\cite{2008ARA&A..46..385F}. 

% For student internship proposals, 
% describe the student or selection process, mentoring strategies, and a summary of the 
% existing internship program if the proposal is to augment that. 

 
We propose to set up a mentoring laboratory for undergraduates to
learn data and model driven LSST science at the Center for Particle
Astrophysics at Fermilab.

The 7$^{th}$ floor at Fermilab's Wilson Hall is the home to a 
group of experimental astrophysicists that once helped build and operate
the revolutionary Sloan Digital Sky Survey, helped build and operate
the evolutionary Dark Energy Survey, and will soon be helping to operate 
the revolutionary Large Synoptic Survey Telescope. Cosmic surveys are
what the group was founded to do in 1992 and what we've been doing for 25 years.

The culture here is interesting: for example, the postdocs do not work for
a supervisor, they are their own independent scientists. Often our job as 
staffers here at the Lab is to support the postdocs in their 
ambitious scientific pursuits. The postdocs 
often produce astounding results- such as Alex Drilca-Wagner's dwarf galaxy searches,
which  enabled new limits on dark matter. Our current set of postdocs 
came to the 7th floor to help us extract science from the DES data stream,
but each are trying to create something new, something of their own or 
something jointly with a colleague. Now they are looking to the future,
looking to perform exciting science  inside the DESC science collaboration.
These postdocs set our science goals: cluster cosmology, dark matter via
dwarf galaxy studies, strong lensing, machine learning.

The culture here also includes mentoring junior scientists.
During the summer the population on the 7th floor more than doubles.
Last year we had six high school students from the Illinois Math
and Science Academy performing student independent research.
We mentored one teacher research assistant from a Texas middle school. 
We mentored 7 undergraduates, and two more
spent several weeks here visiting the lab along with their professor from
Austin Peay State University.  Three graduate students worked with us over the summer,
from Ohio University, University of Ferrara, and the University College London.
The culture of mentoring here is motivated by the idea that these were
collaborations between a  senior scientist and a junior scientist to 
perform a project that otherwise would not be done.

The set of summer 2017 undergraduates was interesting. Four were University of Chicago
undergraduates, one of which had spent three summers at the Lab
and in the fall went to Johns Hopkins for graduate school. 
Two were from the Lab's SIST program: Summer Interns in Science 
and Technology, for under-represented minorities majoring in STEM
fields at 4-year colleges. One was from DOE's SULI program: Science
Undergraduate Laboratory Interns in physics and engineering.

In summer of 2017, something interesting happened.
A SULI undergraduate, a SIST undergraduate, and two IMSA high school students worked
together to build a micro-lensing event simulator tool to be used
on stars in the DES data set using the DES cadence. They used Git
to share code, sat in each other's offices to work problems, and
exchanged plots to prepare posters, talks, and reports. 
They self-reported a fantastic experience: each with their strengths,
they felt part of something greater, part of a team building something 
for science. That is, they felt part of the large collaboration experience.
We believe we should build on this.

We propose to set up a LSST undergraduate internship laboratory
on the 7th floor of Wilson Hall following the lesson
learned from the experience of the micro-lensing group. 
Peer groups learn from each other,
and differences in experience represent chances to teach: high school
students learn readily from undergraduates, and undergraduates learn
by leading high school students. The context was of learning science
via code development and testing; in this context the experiment was a great
success.

\newpage
The way we would build the laboratory is
to hire a post-baccalaureate to bring up to
speed in LSST DESC notebooks, simulated data sets, and the CosmoSIS
cosmological parameter estimation code  before the
summer began.  The role of the
post-baccalaureate is to be a lab manager: showing the summer interns
how to operate in the lab.
During the summer we would bring in two LSST interns
and compete for a Lab SIST and SULI intern.  The lab manager can
teach the new interns how to get up and running on the Lab-provided
laptop computers, how to connect to the 7th floor compute farm,
how to print (important and not straightforward!), and answer what does
it mean to make a plot. All of these simple things a senior
scientists can do, but the price is that the hour per day they have 
available for direct interaction is used for something other than 
the mentoring on how to do science with large data.

What we would ask the interns to do
is to re-create a famous plot in astrophysics and cosmology using
the DC2 proto-catalogs, the CosmoSIS cosmological modeling program,
DESC Jupyter notebooks, Github, and teamwork. Once they have done so,
we ask them to make an improvement. That simple, and that hard.

The actual science we have them do is secondary: perhaps it is the
halo mass function, perhaps it is the galaxy-galaxy correlation function.
All of these are new to undergraduates, and by doing the classics they
become ready to do things new.

This is a change to our previous model of mentoring:
collaborate to perform a science project. We choose to do this
as we intend to prepare the students to become active scientists
with LSST/DESC science products and there the learning curve is greater.
The reason to try this approach is in the students: where they go next.
Last year, one IMSA micro-lenser went to
Northeastern for college, another we get back this summer. The SULI
intern is applying the graduate school; she may well be back at Fermilab
in a year as a collaborator. The SIST intern is transitioning from 
an engineering career to become an astrophysics graduate student- 
she is the ideal candidate
to be the post-baccalaureate we hire with this grant. She can build
up her skill set and astrophysical background before competing for
graduate school entry. Our aim is to build a summer laboratory 
that prepares junior scientists to thrive in the LSST era: there
are several paths to do this. 

To summarize, we  propose to leverage off of the tradition 
of mentoring at Fermilab by
bringing on two undergraduate interns from this funding
and several more from existing Fermilab intern programs.
The interns will form a lab around a more experience
lab manager, and will explore the science
of combining DC2 science analysis with CosmoSIS modeling
under the supervision of experienced cosmic
frontier scientists, a model that was shown to work
well last year. Many of these  undergraduates will
form the candidate graduate student classes in years to come.

\newpage
\section{The implementation of this program}

Fermilab has an active internship program. Two internship programs
have been mentioned, and there are four others. The internship
programs here share a Summer Undergraduate Lecture Series by
Lab and visiting scientists. There is an end of summer poster
session given by the summer interns. All of these interns
are housed near the Lab, some on the Lab; we have institutional
experience in handling housing. Lastly, as the US center for
experimental high energy physics, Fermilab itself inspires
the imagination of its interns: they are working at the frontier
of knowledge.

Our base summer intern selection process is to attract
University of Chicago Physics and Astronomy undergraduates.
The Lab has many scientists on joint appointments with
the University of Chicago. When at the University they find
that there are  many more students
requesting research projects then there is the possibility of
the University to support- we believe this is an opportunity.
Other avenues exist: our User community has many undergraduates
from which we could make invitations, 
and the Lab performs student selection for many of its summer
intern programs.

We have identified the right post-baccalaureate candidate:
our SIST intern from last summer has expressed interest in
doing this as she wishes to continue her astrophysics at 
Fermilab before going on to graduate school. There are
other possible candidates if our first choice declines: 
for example a computer scientist/mathematics student
from IIT is looking to spend a year in data science before
attending graduate school in physics.

We would like to bring two of our funded people to the LSST
Project and Community Workshop in August. This is
very likely to be the lab manager and one of the interns.


